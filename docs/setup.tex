%! TEX program = xelatex
%! TEX root = main.tex
\documentclass[a4paper,11pt]{article}
\usepackage{geometry}
\geometry{left=1.5cm,right=5mm,top=1cm,bottom=1.5cm}
%============================中文支持==============================
%中文自动换行
\XeTeXlinebreaklocale "zh"
\XeTeXlinebreakskip = 0pt plus 1pt
%中文编码支持
\usepackage{xltxtra,fontspec,xunicode}
 %中文支持包
\usepackage{xeCJK}
%书签
\usepackage{CJKutf8}
%1. 文件开头加入\usepackage{pdfpages}
%2. 文件中使用\includepdf{mypdffile.pdf} 插入pdf文件。
\usepackage{pdfpages}
%============================图形以及符号===============================
%避免浮动
\usepackage{float}
%插入图片的宏包
\usepackage{graphicx}
%图标包 typicous
\usepackage{typicons}
%作图
\usepackage{tikz}
\usetikzlibrary{graphs}
\usepackage{pgfplots}
\usepackage{mathtools}
\usepgfplotslibrary{groupplots}
%=======================页面,段落,行距,超链接设置===============================
\usepackage{ulem}  %下划线(uline),波浪线(uwave),双下划线(uuline)
                   %删除线(sout),斜删除线(xout),虚线(dashuline),加点(dotuline)
\usepackage{geometry} %横向布局,用横向布局时,使用\usepackage[landscape]{geometry}
%\usepackage[a3paper]{geometry} %a3纸张大小
%\geometry{left=1cm,right=1cm,top=1.5cm,bottom=1.5cm}
\usepackage{indentfirst} %缩进
%%设置缩进
\setlength{\parindent}{2em}
%设置行距
\linespread{1.6}
%加入超链接,用法为\url{https://www.google.com}或\href{https://www.google.com}{Google}
%这里还必须指定为unicode方式,以使书签正常显示
\usepackage[unicode,colorlinks,linkcolor=black,anchorcolor=blue,citecolor=green]{hyperref}
%========================数学公式==================================
\usepackage{latexsym}
\usepackage{amsmath}                 % AMS LaTeX宏包
\usepackage{amssymb}                 % 用来排版漂亮的数学公式
\usepackage{amsbsy}
\usepackage{amsthm}
\usepackage{amsfonts}
\usepackage{mathrsfs}                % 英文花体字体
\usepackage{bm}                      % 数学公式中的黑斜体
\usepackage{relsize}                 % 调整公式字体大小:\mathsmaller, \mathlarger
\usepackage{caption}                % 浮动图形和表格标题样式
\usepackage{times}
\usepackage{esint} %面积分符号\oiint
\usepackage{mathdots}
\allowdisplaybreaks
%========================字体设置=================================
%中文字体
\setCJKmainfont{YouYuan}
\setCJKsansfont{YaHei Consolas Hybrid}
\setCJKmonofont{YaHei Consolas Hybrid}
\setCJKfamilyfont{SimSun}{SimSun}
\setCJKfamilyfont{SimHei}{SimHei}
\setCJKfamilyfont{FangSong}{FangSong}
\setCJKfamilyfont{KaiTi}{KaiTi}
\setCJKfamilyfont{YouYuan}{YouYuan}
\newcommand{\song}{\CJKfamily{SimSun}} % 宋体
\newcommand{\hei}{\CJKfamily{SimHei}}   % 黑体
\newcommand{\fs}{\CJKfamily{FangSong}}     % 仿宋
\newcommand{\kai}{\CJKfamily{KaiTi}}   % 楷体
\newcommand{\yy}{\CJKfamily{YouYuan}}   % 楷体
%英文字体
\setmainfont{Source Code Pro}
\setmonofont{Source Code Pro}
% 其他英文字体别名
% 使用方法如下:{\sp :g/pattern/s/old/new/g}
\newfontfamily\sp{Segoe Print}

%数学长等号
\usepackage{extarrows}
%特殊符号
\usepackage{bbding}
%矩阵中使用虚线的宏包
\usepackage{pmat}
%=======================页眉页脚设置=======================
%以下两行设置脚注
\renewcommand*{\thefootnote}{\arabic{footnote}} %脚注为罗马数字编号
\usepackage[perpage,stable]{footmisc} % 每页重新编号,如果不希望这样可以去掉 [perpage]

\usepackage{fancyhdr}
\usepackage{pageslts}
%设置subsubsection{}
%\setcounter{secnumdepth}{5}%设置标题深度为5
%subsubsection{} level3
%paragraph{} level4
%subparagraph{} level5
\pagenumbering{arabic} %默认选用的页码为数字格式
%==========================代码格式化包===============================
\usepackage{fancyvrb}
\usepackage{color}
\usepackage[colorlinks,linkcolor=black,anchorcolor=blue,citecolor=green]{hyperref}
\usepackage{listings}

%一般 lstlisitng 样式
\lstdefinestyle{general}{
    basicstyle=\linespread{0.8}\small\ttfamily, % 代码行距设置为 0.8
    frameround=fttt,
    frame=trBL,
    numbers=left,
    firstnumber=1,
    morecomment=[l],
    keepspaces=true,
    morecomment=[s][\color{blue}]{/*+}{*/},
    morecomment=[s][\color{green}]{/*-}{*/},
    morecomment=[s][\color{red}]{/**}{*/},
}

% 模拟 verb 环境实现自动换行
\lstdefinestyle{verb} {
    aboveskip=10pt, % lst 和上面的间距
    belowskip=0pt, % lst 和下面的间距
    basicstyle=\linespread{0.8}\small\ttfamily,  % 代码行距设置为 0.8
    columns=flexible,
    breaklines=true, % 启用自动换行
    frame=single, % 单边框
    postbreak=\mbox{\textcolor{red}{$\hookrightarrow$}\space}, % 换行符号
}

%自定义行间极限符号
\newcommand{\ilim}{\lim\limits}
%=========================自定义目录====================
\renewcommand{\abstractname}{}
\renewcommand{\contentsname}{目录}
\renewcommand{\refname}{参考文献}
\renewcommand{\today}{\number\year 年 \number\month 月 \number\day 日}
% 自定义图片标题的名称
\renewcommand{\figurename}{图}
% 自定义表格标题的名称
\renewcommand{\tablename}{表}
% 自定义代码段标题
\renewcommand{\lstlistingname}{代码}

%自定义向量粗体显示
\renewcommand{\vec}[1]{\boldsymbol{#1}}
%\usepackage{fouriernc}
%改善自定义圆圈数字
%在枚举环境中可以这么做:
%\begin{enumerate}[label=\protect\circled{\arabic*}]
%\item First item
%\item Second item
%\item Third item
%\item Fourth item
%\end{enumerate}
%在单个语句中可以这么做:\circled{1} \circled{2} \circled{3}
\usepackage{enumitem}
% 注意这种圆圈数字有限制,不能在 listing 环境中使用
% 在 listing 中可以使用\textcircled{\raisebox{-0.9pt}{8}}
\newcommand*\circled[1]{\tikz[baseline=(char.base)]{
            \node[shape=circle,draw,inner sep=2pt] (char) {#1};}}
%加载目录以及标题设置宏包
\usepackage{titlesec}
\usepackage{titletoc}
%设置目录中的字体,编号以及边距:
\titlecontents{part}[10mm]                          %设置part的编号距离纸张左侧10mm处开始打印
{\fontsize{13pt}{10pt}\selectfont \song}            %字体设置为12pt,行距设置为10pt,字体设置为宋体
{\contentslabel{2em} }                              %part编号和标题之间的距离为2em
{}
{ \contentsmargin{4mm} \titlerule*{.}\contentspage} %设置标题右侧页码与连接符号之间的距离为4mm,设置标题和页码之间的连接符号为点,并设置间隔距离


\renewcommand\thesection{\arabic{section}}          %解决section从0.x开始的问题
\titlecontents{section}[10mm]                       %设置section的编号距离纸张左侧10mm处开始打印
{\fontsize{12pt}{10pt}\selectfont \song}            %字体设置为12pt,行距设置为10pt,字体设置为宋体
{\contentslabel{2em} }                              %section编号和标题之间的距离为2em
{}
{ \contentsmargin{4mm} \titlerule*{.}\contentspage} %设置标题右侧页码与连接符号之间的距离为4mm,设置标题和页码之间的连接符号为点,并设置间隔距离


\titlecontents{subsection}[22mm]
{\fontsize{11pt}{10pt}\selectfont \song}
{\contentslabel{3em}}
{}
{ \contentsmargin{4mm} \titlerule*{.}\contentspage}

\titlecontents{subsubsection}[37mm]
{\fontsize{10pt}{10pt}\selectfont \song}
{\contentslabel{4em}}
{}
{ \contentsmargin{4mm} \titlerule*{.}\contentspage}
%%设置正文中的标题大小以及字体
\titleformat{\part} {\LARGE\yy} {\thesection}{1em}{\centering}{} %part标题居中显示
\titleformat{\section} {\LARGE\yy} {\thesection}{1em}{}
\titleformat{\subsection} {\Large\yy} {\thesubsection}{1em}{}
\titleformat{\subsubsection} {\large\yy} {\thesubsubsection}{1em}{}

%设置分类图
\newenvironment{subgroup} {$\left\{\tabular{l}} {\endtabular\right.$}
%\begin{subgroup}
%    数模混合计算机 \\[1em]
%    模拟计算机\\[1em]
%    数字计算机
%    \begin{subgroup}
%        专用计算机 \\
%        通用计算机
%        \begin{subgroup}
%            巨型机\\
%            大型机\\
%            小型机\\
%            微型机\\
%            工作站\\
%            服务器
%        \end{subgroup}
%    \end{subgroup} \\[5em]
%    未来计算机
%    \begin{subgroup}
%        量子计算机 \\
%        生物计算机
%    \end{subgroup}
%\end{subgroup}
